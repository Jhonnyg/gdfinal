
\chapter{Guidelines}

Thinking in terms of pure restrictive/unrestrictive guidelines is quite hard with this subject in mind. Manipulating time is a general category that indicates what elements must be included in this style, which provides a large amount of freedom for a developer. In the case of design patterns and mechanics, I have used existing patterns as a starting point, elimintating and including features based on the games included in this study. 

\section{Patterns and Mechanics}

The idea of this concept is to drive a game's style of play by changing the timeline, which brings a new dimension of possibilites in terms of game mechanics, concepts and ideas. Listed below is a collection of patterns and abstract mechanics that can be identified from games that fit the definition stated in section 1.1. These are abstract ideas that are not coupled together with a specific set of rules or restrictions, and can be interpreted quite freely with the only condition being that they are be user-directed and involves manipulating time. 

%. Rather, they are a showcase of different ways to incorporate time manipulation by implying that it must be specifed that something %happens when time is changed, and not what these changes may be. The environment stipulates the effects of this manipulation, as %stated by its set of rules. 

\begin{itemize}
\item{Time Travel} \\ Time travel is the largest category of time manipulation, and includes several important features, 
such as the cause-effect relationship and alternate realities. When including this category, it must be clear that the 
time manipulation is strictly user-driven. The time manipulation in a game such as Assassin's Creed is only related to the historic setting and its story, and not as a consequence of user-directed interaction. 

\item{Time Paradox} \\ The idea of time paradox is also related to causality, by moving in time, 
your actions will propagate to present time, enabling you to perform actions not possible with the initial conditions. 
Consider a game based on time manipulation where, in order to .. [example] 
your present self will have to [do something]. In "The misfortune events of P.B Winterbottom", the time paradox and alternate reality 
patterns is manifested by the main character co-existing in real-time with present versions of himself in order to solve certain tasks.

Examples of games with this mechanic: Chronotron, The Misadventures of P.B Winterbottom

\item{Bullet-Time} \\ Also called "slow motion" is a time manipulating feature that simply allows the player to slow 
down the notion of time in the game, however still linearily forwards in time and not in reverse. 
This change can affect either every element or a subset (e.g. the player only, everything but the player or something else) of the the environment. 
Examples of games with this mechanic: Max Payne, Call of Duty: Modern Warfare 2 and F.E.A.R

\item{Time Stopping} \\ Time Stopping is the mechanic that when used, stops time completely, providing negative or most commonly, positive side-effects. 
Examples of games with this mechanic: Prince of persia series, Megaman 2
\end{itemize}

\section{Balance Game Difficulty}
In terms of difficulty, depending on what level of importance time travel has for the gameplay (e.g. when comparing Max Payne to Braid, it is not essential to use bullet-time to complete Max Payne, but you cannot finish Braid without using the time warp functions), it must be properly balanced to maintain a good flow and fight "player fatigue". 

Due to the nature of time travel, the user should generally be encouraged to plan ahead or backwards how time manipulation should be used. This is the basic conditions for a puzzle game, but also true in other genres however not to the same extent. For example, in Call of Duty Modern Warfare II, bullet-time is automatically enabled at certain locations in the game, causing the player to plan what targets to attack in the same fashion as Max Payne. 

As a developer, you generally want the player to experience the full extent of the game from start to finish, but it is also quite essential that the difficulty of the game must scale well over time. Obviously, scaling and balancing the game play must always be weighted in regards to your target audience, creating complex logical puzzles that requires an inclination to problem solving might perhaps not the best approach when designing a game for pre-teenagers, or children. This further stresses the imporance of a well-planned game design documents with a good use of market research. 

%\section{Maintain Flow}



%Do's and dont's when designing games with this gameplay style. 