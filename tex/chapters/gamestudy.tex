
\chapter{Game Study}
% Study of games that utilize this style.

\section{Overview}

%The four games below were picked mainly as they cover different aspects of time manipulation, which in turn provides a quite different gameplay experience. 

\subsection{Braid}
Indie favourite Braid is perhaps the primary title that has embedded time manipulation in its core gameplay. The game is a 2D puzzle platformer with all the standard ingredients ranging from movement, enemy behaviour and world elements such as cannons, ladders and collectibles. The story is heavily inspired by the plot in the Super Mario bros. games, and is unveiled to the player in the start of every level. That aside, the normal platform game challenges, such as difficult jumps, hard enemies and time limit constraints are in fact the trivial part of this game since you can always rewind your actions to the point of safety. Upon dying, the game enters a paused state, and the user is notified to use the rewind button; simply meaning you simply cannot die. Instead, the real challenges of this game are derived from the nature of the time travel mechanisms, where the rules of the game's universe constitute the odd and sometimes mind-boggling puzzles found in the different worlds. Apart from the basic rewind function, the rules themselves are never fully explained, engaging the player to think outside the box to get through the levels. 

Braid explores several such concepts in a more complex manner than any other game currently on the market. In the first levels of the game, the player is introduced to very basic time manipulation. As the game progress, more and more levels of complexity emerge, leading to the final stage of the game where time initially moves backwards. Thus, when using the time warping ability, the time is instead moving fowards.

Braid is a perfect example when it comes to utilizing this idea to the fullest, and in a certain sense, it makes Braid a genre defying title that perhaps expands the realms of what is possible in future gameplay. 

\subsection{The Misadventures of P.B Winterbottom}

Still in production, The Misadventures of P.B Winterbottom mixes time manipulation with fantastic aesthetics, inspired by silent movies from the 1920s. 
The game is a standard side-view 2D puzzle platformer, with a very simple story; P.A Winterbottom, an old british man with an obscure fondness of pie and tall hats,
 is strolling about in a steampunk-esque city, stealing every pie he encounters. One night he finds himself chasing a pie that can bend time and space,
 allowing himself to create and control cloned versions of himself in time. This, and a nifty umbrella that he uses to glide through the air or whack his clones, 
enables his to steal those hard-to-reach pies located on the ten levels that ships with the game.

The time travelling in P.B Winterbottom, is visually and mechanically manifested by "recording" your actions. In order to access a pillar higher than you can jump, 
you can walk over to the base of the pillar and create a clone of yourself. This will enable you to jump from the hat of your previous self, or rather, the recording of 
yourself walking towards the pillar. This is one of the fundamental time-recordings of the game that the player wil need too use, but it is not the only one. You can trigger your clones to perform tasks in time, enabling you to multitask with yourself in real-time.

The mechanics of The Misadventures of P.B Winterbottom are similar to those in braid, but focus entirely on the concept of time paradoxes and explores what can be done in that context. This limitation provides a quite different type of gameplay. After the player has been introduced to the mechanics of time travel, he must use this knowledge as a base for approaching the puzzles through out the game even though the difficulty is increased. This is a distinguishin factor between this game and Braid. Where in Braid, the player is given a new tool of time travel in every new world, and must figure out how to use this mechanic in order to move on. 

% http://www.1up.com/do/reviewPage?cId=3177999

\subsection{Prince of Persia - Sands of Time}
As the title suggests, the gameplay of Prince of Persia - Sands of Time is heavily based on time manipulation, as well as acrobatics and good old sword fighting. 
In sands of time, the player can use a magical item, called the dagger of time, that rewinds time 10 seconds backwards, effectively reversing all actions and events 
that took place during this time. Any damage inflicted upon the player, NPCs and various structures will consequently be restored to the state they were in 10 seconds ago.

Additinonally, the dagger of time also enables the player to slow down time completely, transforming his enemies into frozen targets which are easier to take down. These two main 
functions gives the word freedom a whole new meaning. Anyone who has played the predecessors on the Amiga game system will tell you that such new inventions are 
welcomed features as most of the time, you would find yourself unavoidently falling down on spikes or from really high platforms, always leading to certain death. 
This caused the player to constantly replay several areas until he or she could overcome the obstacle (as I recall, once you died, you had to start all over again from the
 beginning of the level, which ultimately lead to me hating the game forever). Thus, with the introduction of time manipulation fourteen years later, the game received a much appreciated injection and a reboot of the series. 

Rewinding time is a very powerful tool, and must consequently be restricted. The dagger of time can only be used a specific number of times before it 
has to be refilled by sand from another old relic acquired in the game; the Hourglass or Time. This is achieved by absorbing enemies or sand clouds found 
throughout the world. This restriction gives the player an incitament to defeat enemies instead of avoiding them. Stopping time also gives the players more variety in combat, 

\subsection{Max Payne series}

The Max Payne series (I and II subsequently) is a sucessful third-person shooter franchise that feature elements of puzzle solving and acrobratic stunts, although most of the gameplay is concerned with gunfights against several enemies at the same time. In these games, the player can utilize a "The Matrix"-inflouencesd bullet-time mechanic that sets the game to a slow-motion state, enabling the user to perform any action he normally would.
Since the player can still move the camera in real-time and has more time to spot his enemies and plan his actions more carefully,
he now has a great advantage over his enemies, which makes overwhelming situations easier to manage.

Regarding the same arguments as with Prince of Persia, this mechanic is very powerful and must thus be restricted. However, this game does not require special items to replenish this functionality; instead, it is renewed at a static frequency.


%Winterbottoms lulseri
%TODO: För specifikt? Mer generell analys mayhabps :(

%Aaccording to many reviews, The Misadventures of P.B Winterbottom is generally received very well and has achieved overall good ratings. Having read several reviews and without testing the full game, it seems to me that the game generally is 

%, it is quite clear that the game lacks a well-balanced difficulty progression[??]. 
%
%most reviews generally feel that the game is too easy,
%and that a user can generally "power through" all the levels, removing some of the more intensive problem solving, you normally would expect from a puzzle game. Since you can
%redo any record of yourself that doesnt do what you'd expect, it removes some of the difficulty of the game.
