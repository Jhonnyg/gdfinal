\chapter{Definition}

% upplägg
% definition + flytande description? bara definition + flytande i guidelines? 

\section{Concept: User-driven Time manipulation}

% en mening??

Time manipulation is a design concept that covers a strict selection of several game mechanics and patterns. 

This concept is itself not a specific set of rules a developer instantiates in order to achieve this game play style, 
but rather as an activity in which time is perceived as a game element that the user can interact with. 
The word interaction is a the key to this idea, it does not cover any part of a game's story(character x travels through time to kill character y),
 it's settings (alternate reality,historical locations etc.), the life and death of a character (games where characters have multiple lives,
 since dying and then placing this caracter in a predetermined location could be seen as time-travel). With this in mind, the concept 

% time itself is being manipulated, 

% alternate reality

The core idea of user driven time manipulation is that 

The term user-driven is in direct relation to the feedback loop generated between the player and the environment (game state).
 In most games, most of the player-environment interaction is being held inside of the environment itself. The player, manifested by an agent (avatar), multiple agents or other various elements in the game world, does not engage with the environment itself, but acts merely as input to the game. 

Patterns, concepts

% -> analys: ?!?!?
These are some of the already existing patterns,concepts and mechanics that fit this category to a certain extent. 
These are abstract ideas not coupled together with rules or restrictions, if time is manipulated by the user in some way, it is not described 
what will happen but only that something will happen. It is up to the developer to stipulate what will be possible in the environment of the game. 

\begin{itemize}
\item{Time Travel} \\ Time travel is the largest category of time manipulation, and includes several important features, 
such as the cause-effect relationship and alternate realities. When including this category, it must be clear that the time manipulation is strictly user-driven. 
A game such as Assassin's Creed revolves around the use of time travelling as it mostly takes place in historic locations. However, in this case, the
 time manipulation is only related of the story and the setting of the game, not a consequence of user-directed interaction, which is the premise for this definition. 
%aoeu skriv om..

\item{Time Paradox} \\ The idea of time paradox is also related to causality, by moving in time, your actions will propagate to present time,
 enabling you to perform actions previously not possible. Consider a game based on time manipulation where, in order to .. [example] 
your present self will have to [do something]. In "The misfortune events of P.B Winterbottom", the time paradox and alternate reality 
patterns is manifested by the main character co-existing in real-time with present versions of himself to satisfy the winning conditions. 
Examples of games with this mechanic: Chronotron, The Misadventures of P.B Winterbottom

\item{Bullet-Time} \\ Also called "slow motion" is a time manipulating feature that simply allows the player to slow down the notion of time in the game. 
This change can affect either everything in the environment or only a selection (e.g. the player only). 
Examples of games with this mechanic: Max Payne, Call of Duty: Modern Warfare 2 and F.E.A.R

\item{Time Stopping} \\ hey now
Examples of games with this mechanic: Prince of persia series, Megaman 2
\end{itemize}
%Describe the gameplay mechanic in detail