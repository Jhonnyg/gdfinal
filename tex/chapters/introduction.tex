\chapter{User-driven Time Manipulation}

\section{Introduction}

For this assignment, i have chosen to study the use of manipulating time as a gameplay style. The essential part is that the player has the ability to alter the timeline of the game, where in some cases, this is crucial in order to satisfy the winning conditions. Either case, it is concidered as an asset to the player, and can be controlled by user input as well as the game environment. Actions that the player invokes on the timeline will follow [ordval] the philosophy of causality, commonly known as a cause-and-effect relationship between events. 

This concept plays a major part in a large variety of games in different genres, such as Max Payne (First-Person Shooter) and Prince of Persia - Sands of Time (Action Adventure). In this document, both these games, as well as other interesting time-manipulating titles will be covered and discussed. 

\section{Definition}

User-driven Time Manipulation:

\emph{"The style of game play in which time is integrated into the environment as an affectable element, it produces favorable side-effects to the outcome of the game and is governed by user input." }

\section{Description}

This concept in itself does not constitute a specific set of rules that a developer instantiates in order to achieve this gameplay style, 
but rather as an activity in which time is perceived as a game element that the user can interact with. It categorizes certain traits that time manipulating games have in common, as well as [something something]. 
The word interaction is the most significant part here. Otherwise, any part of a game's story(character x travels 
through time to kill character y), it's settings (alternate reality,historical locations etc.), the life and death of a 
character (games where characters have multiple lives, since dying and then placing this caracter in a predetermined 
location could be seen as time-travel) would be included, and it would not provide a reasonable limitation. 

In most games, the player-environment interaction is handled through manifestations of the player, such as an agent/avatar, multiple agents or various other game elements inside the game's environment. This chain of events can be seen as a feedback loop that is fed input from the user then processes this input which manipulates the environment (game state) in some way, and generates a response back as an output.

MORE TO COME!!!
 
