\chapter{User-driven Time Manipulation}

\section{Introduction}

For this assignment, i have chosen to study the use of manipulating time as a gameplay style, where the essential part is that the player has the ability to alter the timeline of the game. In some cases, this is crucial for the player in order to satisfy the winning conditions. 

%skriva om: 
The player 
Every action the player invokes on the timeline will follow [ordval] the philosophy of causality, commonly known as a cause-and-effect relationship between events. 

This mechanic has been used in a large variety of games in different genres, such as Max Payne (First-Person Shooter) and Prince of Persia - Sands of Time (Action Adventure). In this document, both these games as well as other interesting time-manipulating titles will be covered in detail. 

\section{Definition}
User-driven Time Manipulation: The style of game play in which time is integrated into the environment as an affectable element, it produces favorable side-effects to the outcome of the game and is strictly governed by user input. 

\section{Description}

This concept is itself not a specific set of rules a developer instantiates in order to achieve this game play style, 
but rather as an activity in which time is perceived as a game element that the user can interact with. 
The word interaction is a the key to this idea, it does not cover any part of a game's story(character x travels 
through time to kill character y),  it's settings (alternate reality,historical locations etc.), the life and death of a 
character (games where characters have multiple lives, since dying and then placing this caracter in a predetermined 
location could be seen as time-travel). 
%With this in mind, the concept 

The term "user-driven" is the key ingredient here,

 and is directly linked 

It is in direct relation to the feedback loop generated 
between the player and the environment (game state).
 In most games, the player-environment interaction is held inside of the environment itself. The player, manifested by an agent (avatar), multiple agents or other various elements in the game world, does not engage with the environment itself, but acts merely as input to the game. 
