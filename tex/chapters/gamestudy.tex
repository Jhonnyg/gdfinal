
\chapter{Game Study}
% Study of games that utilize this style.

\section{Overview}

\subsection{Braid}
Indie favourite Braid, is perhaps nowadays the primary title that has embedded time manipulation in it's core gameplay. The game is your standard 2D puzzle platformer with all the standard ingredients, ranging from movement, enemy behaviour and world elements such as cannons, ladders and collectibles. The story, heavily "inspired" by is unveiled in the start of every map, and is although somewhat interesting, it plays little part in this game. 

all the basic super-mario type game mechanics: the player can move from side-to-side, jump 

The normal platform game challenges, such as difficult jumps, hard enemies and blablabla are in fact the trivial part of this game, due to the fact that you can always rewind your actions to the point of safety. Instead, the real hard part are derived directly from the nature of the time travel mechanisms, where the rules of the game's universe constitute the odd and sometimes mind-boggling puzzles found in the different worlds. Apart from the basic rewind function, the rules themselves are never fully explained, engaging the player to think outside the box to get through the levels. 

Braid explores the concept of time manipulation in a whole more complex manner than any other game currently on the market. In the first levels of the game, the player is introduced to very basic time manipulation 

\subsection{The Misadventures of P.B Winterbottom}

Still in production, The Misadventures of P.B Winterbottom mixes time manipulation with fantastic aesthetics, inspired by silent movies from the 1920s. 
The game is a standard side-view 2D puzzle platformer, with a very simple story; P.A Winterbottom, an old british man with an obscure fondness of pie and tall hats,
 is strolling about in a steampunk-esque city, stealing every pie he encounters. One night he finds himself chasing a pie that can bend time and space,
 allowing himself to create and control cloned versions of himself in time. This, and a nifty umbrella that he uses to glide through the air or whack his clones, 
enables his to steal those hard-to-reach pies located on the ten levels that ships with the game.

The time travelling in P.B Winterbottom, is visually and mechanically manifested by "recording" your actions. In order to access a pillar higher than you can jump, 
you can walk over to the base of the pillar and create a clone of yourself. This will enable you to jump from the hat of your previous self, or rather, the recording of 
yourself walking towards the pillar. This is one of the fundamental time-recordings of the game that the player wil need too use, but it is not the only one. You can trigger your clones to shoot you through 

% http://www.1up.com/do/reviewPage?cId=3177999

The time-travelling mechanisms are similar to those in braid. 

\subsection{Prince of Persia - Sands of Time}
As the title suggests, the gameplay of Prince of Persia - Sands of Time is heavily based on time manipulation, as well as acrobatics and good old sword fighting. 
In sands of time, the player can use a magical item, called the dagger of time, that rewinds time 10 seconds backwards, effectively reversing all actions and events 
that took place during this time. Any damage inflicted upon the player, npcs and various structures will consequently be restored to the state they were in 10 seconds ago.
 Additinonally, the dagger of time also enables the player to slow down time completely, transforming his enemies into frozen targets, easier to take down. These two main 
functions gives the word freedom a whole new meaning. Anyone who has played the predecessors on the amiga game system will tell you that such new inventions are 
welcomed features. Most of the time, you would find yourself unavoidently falling down on spikes or from really high platforms, always leading to certain death. 
This caused the player to constantly replay several areas until he or she could overcome the obstacle (as I recall, once you died, you had to start all over again from the
 beginning of the level, which ultimately lead to me hating the game for all eternity). Thus, with the introduction of time manipulation fourteen years later, the game received a much appreciated injection. 

Rewinding time is a very powerful tool, and must consequently be restricted. The dagger of time can only be used a specific number of times before it 
has to be refilled by sand from another old relic acquired in the game; the Hourglass or Time. This is achieved by absorbing enemies or sand clouds found 
throughout the world. This restriction gives the player an incitament to defeat enemies instead of avoiding them. Stopping time also gives the players more variety in combat, 

In Sands of time, the player controls the main character in a 3D environment 

\subsection{Max Payne series}
The First-Person Shooter (FPS) series of Max Payne features time manipulation in a simpler fashion. In this series (Max Payne I and II), 
the player can utilize a bullet-time mechanic which slows down the game time in a slow-motion type behaviour, enabling the user to perform any action he normally would.
Since the player can still move the camera in real-time and has more time to spot his enemies and plan his actions more carefully, 
he now has a great advantage over his enemies, which makes otherwise near impossible situations easier to manage. 


\section{Analysis}
%Winterbottoms lulseri
Aaccording to many reviews, the game is generally received very well and has achieved overall good ratings. Having read several reviews and without testing the full game, it seems to me that the game generally is 

, it is quite clear that the game lacks a well-balanced difficulty progression[??]. 

most reviews generally feel that the game is too easy,
 and that a user can generally "power through" all the levels, removing some of the more intensive problem solving, you normally would expect from a puzzle game. Since you can
 redo any record of yourself that doesnt do what you'd expect, it removes some of the difficulty of the game.

% balancing
Even though as a developer, you generally want the player to experience the full extent of the game from start to finish, but it is also quite essential
 that the difficulty of the game must scale well over time. Obviously, scaling and balancing the game play must always be weighted in regards to your target audience,
 creating complex logical puzzles that requires you to plan ahead is perhaps not the best approach when designing a game for pre-teenagers, or children. 
This further stresses[?] the imporance of a well-planned game design documents with proper market research [??? fan vet jag..]. 

With the nature of this type of time travel, the user should generally be encouraged to plan ahead, or rather backwards, in order to solve the puzzles the game challenges you with. 

% Forza

% The first installment of this series was released in July 2001, right in the midst of the Matrix Trilogy where bullet-time was first developed. 

%\section{Chronotron}